\chapter{Summary and Future Work}\label{ch:summary}
%\thispagestyle{empty}

\runningchaptertitle{Summary and Future Work}
%% Use this code to put your background image
%\ThisCenterWallPaper{\wpScaling}{ChapterBackground.png}
%\noabstract
\ThumbIndexShow


High-resolution CT is an important modality to non-invasively diagnose pulmonary diseases and assess treatment effects. In this thesis, we developed automatic methods to quantify pulmonary vasculature and assess treatment effects of CTEPH disease, based on high-resolution CT images. Within an HRCT scan, pulmonary vessels are automatically extracted with a graph-cuts based method, and subsequently the extracted pulmonary vessels are objectively quantified with quantification methods. In this chapter, we summarize the previous chapters and discuss interesting directions of future research.

\section{Summary}

In this thesis, we first provided a general introduction in Chapter 1 about pulmonary anatomy, diseases, clinical assessments, and chest CT scans. A lung vessel segmentation method was proposed in Chapter 2, as accurately extracting lung vessels is an essential step for pulmonary vessel analysis. An automatic method for pulmonary vessel quantification was developed in Chapter 3, where two imaging biomarkers $\alpha$ and $\beta$ were proposed for quantifying the vascular morphology. In Chapter 4, the relation between these imaging biomarkers and pulmonary function were investigated in a selected SSc patient group, who had reductions in gas transfer but did not have fibrosis. The densitometry changes in pulmonary vasculature and parenchyma were studied by comparing CTPA scans before and after treatment for CTEPH patients treated with BPA, in Chapter 5. A vascular tree matching method was proposed for matching pulmonary vasculature trees, for quantifying changes in vascular morphology for CTEPH patients in Chapter 6.  

\textbf{Chapter 2} For lung CT image analysis, lung vessel segmentation is an important processing step. Filters that are based on analyzing the eigenvalues of the Hessian matrix are popular for enhancing pulmonary vessels. However, due to their low response at vessel bifurcations and vessel boundaries, extracting lung vessels by thresholding the vesselness response is not sufficiently accurate. The graph-cuts method could provide more accurate segmentations, as it considers neighborhood information when determining the label of a voxel. We propose a new graph-cuts based method, where the appearance (CT intensity) and shape (vesselness) are combined into one cost function. As the number of voxels in high-resolution CT image is large, building the corresponding graph structure requires a lot of memory and is time consuming. Therefore, an efficient and low memory cost strategy is proposed for constructing the graph structure. Then, the lung vessels are segmented by minimizing the energy cost function with the graph-cuts optimization framework, where the energy cost function is calculated based on the constructed graph. The proposed method is trained and validated by an in-house data set, and independently evaluated with a public data set of the VESSEL12 challenge. According to the evaluation results, the proposed method is accurate, and obtains a competitive performance in VESSEL12.         

\textbf{Chapter 3} Pulmonary vascular remodeling is a significant pathological feature of various pulmonary diseases. In this chapter, we propose an automatic method for quantifying pulmonary vascular morphology in CT images. There are two processing steps in the proposed method: pulmonary vessel extraction and vessel quantification. The vessels are extracted with an improved graph-cuts based method, which incorporates the appearance (CT intensity) and shape features (vesselness from a Hessian-based filter), and considers distance to airways into the cost function. For quantifying the extracted pulmonary vessels, a radius histogram is generated by counting the occurrence of vessel radii, calculated from a distance transform based method. Subsequently, two biomarkers, slope $\alpha$ and intercept $\beta$, are calculated by linear regression on the radius histogram. A public data set of VESSEL12, a data set of 3D printed vessel phantom and a clinical data set of scleroderma patients are involved for evaluating and validating the proposed method. Based on the results, the proposed method is highly accurate, by validating with a public data set and a 3D printed vessel phantom data set. The correlation between imaging biomarkers and diffusion capacity in clinical data confirms an association between lung structure and function.

\textbf{Chapter 4} For systemic sclerosis (SSc), gas transfer is known to be affected by fibrotic changes in the pulmonary parenchyma. However, SSc patients without detectable fibrosis can still have impaired gas exchange. We investigate the pulmonary vascular changes of a patient group in SSc without fibrosis, where remodeling of pulmonary vasculature may partly explain the reduction of gas transfer. Seventy-seven SSc patients were selected who underwent pulmonary function tests and CT scanning, that showed no visible fibrosis. The pulmonary vascular morphology was quantified into two imaging biomarkers, with a primary method to the one in Chapter 3. The association between imaging biomarkers and gas transfer (DLCOc \%predicted) was investigated, which showed a moderate but significant correlation between pulmonary vascular morphology and gas transfer. In conclusion, in SSc patients without pulmonary fibrosis, impaired gas exchange is associated with alterations in pulmonary vascular morphology. 

\textbf{Chapter 5} Patients with inoperable CTPEH, BPA can be an alternative treatment to improve the clinical status and hemodynamics. The invasive right heart catheterization serves as gold standard in evaluating the severity and assessing the treatment effects of CTEPH. In this chapter, we proposed an objective and automatic method to non-invasively assess treatment effects, by comparatively analyzing CTPA of pre- and post- BPA treatment. A cohort of 14 patients in CTEPH, who underwent both CTPA and RHC, before and after BPA, are involved in this study. The densitometric changes in pulmonary vessels and parenchyma are automatically quantified, where the vessels and parenchyma are separated by the graph-cuts based method. The association between perfusion changes and hemodynamic changes are investigated, where the densitometric parameter are significantly correlated with RHC measurements. Based on the CTPA, quantifying the perfusion changes provides non-invasive measures that reflect hemodynamic changes.

In \textbf{Chapter 6}, we propose a pulmonary vessel tree matching method, which enables the quantification of pulmonary morphological changes, longitudinally. In the proposed method, first, the pulmonary vessels are simplified and constructed into a directed graph, a vascular tree is structured by stripping cyclic edges, which makes the quantification at the branch level possible. Then, a tree matching method is proposed, by considering a geodesic path for local topology preservation. In the last processing step, the resistance changes of each branch are analyzed, based on the Poiseulle’s law. Two datasets, a synthetic data set and a clinical data set, are used to validate the accuracy and clinical relevance of the proposed method, respectively. In the results, the proposed vessel tree matching method performs better than two exist methods, and the resistance changes of pulmonary vessels is correlated with hemodynamic changes.

\section{Future Work}

The work presented in this thesis was aimed at developing methods to quantify pulmonary vessels, based on the CT images. The entire pulmonary vascular trees, both arteries and veins, are automatically extracted and subsequently investigated. It is known, however, that pulmonary diseases may affect arteries or veins in a different way. Investigating the pulmonary arteries and veins separately will be of great help in evaluating pulmonary diseases. Separating and classifying pulmonary arteries and veins is, however, a challenging project. With the classification power of deep learning, it would be possible to develop deep-learning based methods to separate arteries and veins. By preparing a large amount of manually annotated data for training, developing deep learning based method for separating pulmonary arteries and veins would be an interesting topic for future. If the pulmonary arteries and veins are successfully separated, there will be multiple interesting clinical applications, such as separately quantifying morphological changes of pulmonary arteries or veins of patients in SSc, quantifying the A/V perfusion changes of BPA treatment for patients in CTEPH, etc.

For the studies on SSc (Chapter 3 and 4), only baseline CT images of patients were investigated. Exploring vascular morphology longitudinally will also be an interesting future topic. According to the protocol of the biobank of the Leiden Combined Care in SSc (CISS), patients are scanned at both full-inspiration and full-expiration level. For patients with SSc, the elasticity of lungs can be influenced significantly by pulmonary fibrosis. Developing methods, by mapping the inspired and expired CT scans, may provide a way to investigate the elasticity of pulmonary parenchyma, which may help to evaluate the severity of SSc-related pulmonary diseases.

\section{General conclusions}
In conclusion, this thesis proposes automatic methods for quantifying pulmonary vessels. An accurate and well-validated lung vessel segmentation method is developed and open source online. Investigating changes in pulmonary vascular morphology may be helpful in understanding the pathophysiology of the SSc patients whose gas transfer deteriorates in the course of their disease without detectable pulmonary fibrosis. Assessing the perfusion changes in pulmonary vasculature using automatic comparison of CTPAs acquired before and after treatment could reflect the hemodynamic changes. Quantifying pulmonary vascular changes in morphology or densitometry may provide non-invasively diagnosis of pulmonary diseases and assessments of treatment effects, based on CT.

 
